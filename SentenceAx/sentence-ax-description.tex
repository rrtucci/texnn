\documentclass[12pt]{article}
\usepackage{graphicx}     %Note graphics<graphicx<epsfig
\usepackage[table,xcdraw, dvipsnames]{xcolor}
\usepackage{hyperref} %written by Sebastian Rahtz
%\usepackage{subeqn}    %allows equations 1a, 1b
%\usepackage{subfig}    %allows figures 1a, 1b
\usepackage{amsmath}
\usepackage{amssymb}
\usepackage{xurl}
\usepackage{bbm}%for indicator function
%\bibliographystyle{alpha}
\usepackage{floatrow}


\usepackage[toc,page]{appendix}
\usepackage[nottoc]{tocbibind}

\usepackage[color,matrix,frame,arrow,curve]{xy}
\usepackage[shortlabels]{enumitem}
\usepackage{array}
\usepackage{pdflscape}
\usepackage{multicol, multirow}
\usepackage{ctable}
\usepackage{dirtree}
\usepackage{subfig}
\usepackage{longtable}

%\usepackage{fdsymbol}
%\usepackage{animate}% play animation

\usepackage[vlined,ruled]{algorithm2e}
\newcommand{\mycommfont}[1]{\small\ttfamily\textcolor{blue}{#1}}
\SetCommentSty{mycommfont}

%highlighting
\usepackage{mdframed}

\usepackage[makeroom]{cancel}

\usepackage{pdfpages}
\usepackage{hyperxmp}

\paperheight=11in
\topmargin=0in
\headheight=0in
\headsep=0in
\topskip=0in
\textheight=8.5in
\footskip=.5in


\paperwidth=8.5in
\oddsidemargin=.25in
\evensidemargin=.25in
\textwidth=6.0in
\parindent=.5in

\newcommand{\chapquote}[3]{\begin{quotation} \textit{#1} \end{quotation} \begin{flushright} - #2, \textit{#3}\end{flushright} }

\newtheorem{claim}{Claim}
\newcommand{\proof}[0]{{\bf proof:} }
\newcommand{\qed}[0]{\quad\newline \noindent{\bf QED }}


\newcommand{\bra}[1]{\langle#1|}
\newcommand{\ket}[1]{|#1\rangle}
\newcommand{\av}[1]{\left\langle#1\right\rangle}
\newcommand{\pder}[2]{\frac{\partial#1}{\partial#2}}
\newcommand{\der}[2]{\frac{d#1}{d#2}}
\newcommand{\tr}[0]{{\rm tr }}
\newcommand{\beq}{\begin{equation}}
\newcommand{\eeq}{\end{equation}}
\newcommand{\bsub}{\begin{subequations}}
	\newcommand{\esub}{\end{subequations}}
\newcommand{\beqa}{\begin{eqnarray}}
\newcommand{\eeqa}{\end{eqnarray}}
\newcommand{\rarrow}[0]{\rightarrow}
\newcommand{\larrow}[0]{\leftarrow}
\newcommand{\uarrow}[0]{\uparrow}
\newcommand{\darrow}[0]{\downarrow}
\newcommand{\Rarrow}[0]{\Rightarrow}
\newcommand{\nRarrow}[0]{\nRightarrow}
\newcommand{\Larrow}[0]{\Leftarrow}
\newcommand{\nLarrow}[0]{\nLeftarrow}
\newcommand{\ul}[1]{\underline{#1}}
\newcommand{\ol}[1]{\overline{#1}}
\newcommand{\WW}[0]{ {\mathbb{W}}}
\newcommand{\ZZ}[0]{ {\mathbb{Z}}}
\newcommand{\RR}[0]{{ \mathbb{R}} }
\newcommand{\CC}[0]{{ \mathbb{C}} }
\newcommand{\XX}[0]{{ \mathbb{X}} }
\newcommand{\ground}{{}_{\stackrel{\stackrel{\displaystyle{\bot}}{-}}{.}}}
\newcommand{\norm}[1]{\parallel#1\parallel}
\newcommand{\eqdef}[0]{\;\;\stackrel{\text{def}}{=}\;\;}

\newcommand{\heavy}[0]{\mathbf{H}}

\newcommand{\rva}[0]{{\ul{a}}}
\newcommand{\rvb}[0]{{\ul{b}}}
\newcommand{\rvc}[0]{{\ul{c}}}
\newcommand{\rvd}[0]{{\ul{d}}}
\newcommand{\rve}[0]{{\ul{e}}}
\newcommand{\rvf}[0]{{\ul{f}}}
\newcommand{\rvg}[0]{{\ul{g}}}
\newcommand{\rvh}[0]{{\ul{h}}}
\newcommand{\rvi}[0]{{\ul{i}}}
\newcommand{\rvj}[0]{{\ul{j}}}
\newcommand{\rvk}[0]{{\ul{k}}}
\newcommand{\rvl}[0]{{\ul{l}}}
\newcommand{\rvll}[0]{{\ul{\ell}}}
\newcommand{\rvm}[0]{{\ul{m}}}
\newcommand{\rvn}[0]{{\ul{n}}}
\newcommand{\rvo}[0]{{\ul{o}}}
\newcommand{\rvp}[0]{{\ul{p}}}
\newcommand{\rvq}[0]{{\ul{q}}}
\newcommand{\rvr}[0]{{\ul{r}}}
\newcommand{\rvs}[0]{{\ul{s}}}
\newcommand{\rvt}[0]{{\ul{t}}}
\newcommand{\rvu}[0]{{\ul{u}}}
\newcommand{\rvv}[0]{{\ul{v}}}
\newcommand{\rvw}[0]{{\ul{w}}}
\newcommand{\rvx}[0]{{\ul{x}}}
\newcommand{\rvy}[0]{{\ul{y}}}
\newcommand{\rvz}[0]{{\ul{z}}}


\newcommand{\rvA}[0]{{\ul{A}}}
\newcommand{\rvB}[0]{{\ul{B}}}
\newcommand{\rvC}[0]{{\ul{C}}}
\newcommand{\rvD}[0]{{\ul{D}}}
\newcommand{\rvE}[0]{{\ul{E}}}
\newcommand{\rvF}[0]{{\ul{F}}}
\newcommand{\rvH}[0]{{\ul{H}}}
\newcommand{\rvK}[0]{{\ul{K}}}
\newcommand{\rvL}[0]{{\ul{L}}}
\newcommand{\rvN}[0]{{\ul{N}}}
\newcommand{\rvQ}[0]{{\ul{Q}}}
\newcommand{\rvR}[0]{{\ul{R}}}
\newcommand{\rvT}[0]{{\ul{T}}}
\newcommand{\rvU}[0]{{\ul{U}}}
\newcommand{\rvV}[0]{{\ul{V}}}
\newcommand{\rvW}[0]{{\ul{W}}}
\newcommand{\rvX}[0]{{\ul{X}}}
\newcommand{\rvY}[0]{{\ul{Y}}}
\newcommand{\rvZ}[0]{{\ul{Z}}}

\newcommand{\rvxi}[0]{{\ul{\xi}}}
\newcommand{\rvzeta}[0]{{\ul{\zeta}}}
\newcommand{\rvtheta}[0]{{\ul{\theta}}}
\newcommand{\rvmu}[0]{{\ul{\mu}}}
\newcommand{\rvsig}[0]{{\ul{\sigma}}}
\newcommand{\rvDel}[0]{{\ul{\Delta}}}
\newcommand{\rvtau}[0]{{\ul{\tau}}}

\newcommand{\cala}[0]{{\cal A}}
\newcommand{\calb}[0]{{\cal B}}
\newcommand{\calc}[0]{{\cal C}}
\newcommand{\cald}[0]{{\cal D}}
\newcommand{\cale}[0]{{\cal E}}
\newcommand{\calf}[0]{{\cal F}}
\newcommand{\calg}[0]{{\cal G}}
\newcommand{\calh}[0]{{\cal H}}
\newcommand{\cali}[0]{{\cal I}}
\newcommand{\calk}[0]{{\cal K}}
\newcommand{\call}[0]{{\cal L}}
\newcommand{\calm}[0]{{\cal M}}
\newcommand{\caln}[0]{{\cal N}}
\newcommand{\calo}[0]{{\cal O}}
\newcommand{\calp}[0]{{\cal P}}
\newcommand{\calq}[0]{{\cal Q}}
\newcommand{\calr}[0]{{\cal R}}
\newcommand{\cals}[0]{{\cal S}}
\newcommand{\calt}[0]{{\cal T}}
\newcommand{\calu}[0]{{\cal U}}
\newcommand{\calv}[0]{{\cal V}}
\newcommand{\calw}[0]{{\cal W}}
\newcommand{\calx}[0]{{\cal X}}
\newcommand{\caly}[0]{{\cal Y}}
\newcommand{\calz}[0]{{\cal Z}}

\newcommand{\Pmat}[4]{\calp\left[
\begin{array}{cc}#1&#2\\#3&#4
\end{array}\right]}

%\newcommand{\PN}[0]{PN^{0,0}_{1,1}}
%\newcommand{\PS}[0]{PS^{1,1}_{0,0}}
\newcommand{\PN}[0]{PN}
\newcommand{\PS}[0]{PS}

\newcommand{\lam}[0]{\lambda}
\newcommand{\Lam}[0]{\Lambda}
\newcommand{\alp}[0]{\alpha}
\newcommand{\eps}[0]{\epsilon}
\newcommand{\s}[0]{\sigma}
\newcommand{\su}[0]{{\Sigma}}

\newcommand{\pp}[0]{\mathbb{P}}
\newcommand{\dbm}[0]{{
		[1,\partial_{b},\partial_{m}]
}}

\newcommand{\dg}[0]{{[1, \partial_{\theta_G}]}}
\newcommand{\dd}[0]{{[1, \partial_{\theta_D}]}}
\newcommand{\dgd}[0]{{[1, \partial_{\theta_G}, \partial_{\theta_D}]}}

\newcommand{\veca}[0]{{\vec{a}}}
\newcommand{\vecb}[0]{{\vec{b}}}
\newcommand{\vecc}[0]{{\vec{c}}}
\newcommand{\vecd}[0]{{\vec{d}}}
\newcommand{\vecf}[0]{{\vec{f}}}
\newcommand{\vech}[0]{{\vec{h}}}
\newcommand{\vecr}[0]{{\vec{r}}}
\newcommand{\vecs}[0]{{\vec{s}}}
\newcommand{\vecu}[0]{{\vec{u}}}
\newcommand{\vecx}[0]{{\vec{x}}}
\newcommand{\vecy}[0]{{\vec{y}}}
\newcommand{\vechy}[0]{{\vec{\haty}}}
\newcommand{\vtheta}[0]{{\vec{\theta}}}

\newcommand{\haty}[0]{{\widehat{y}}}
\newcommand{\hatx}[0]{{\widehat{x}}}
\newcommand{\hata}[0]{{\widehat{a}}}
\newcommand{\hatr}[0]{{\widehat{r}}}


\newcommand{\cond}[0]{{\:\mathbf{|}\:}}
\newcommand{\mymathbf}[1]{#1}

\newcommand{\ranvec}[1]{\ul{\vec{#1}}}
\newcommand{\indi}[0]{\mathbbm{1}}
\newcommand{\smoid}[0]{{\rm smoid}}
\newcommand{\lodds}[0]{{\rm lodds}}
\newcommand{\expit}[0]{{\rm expit}}
\newcommand{\logit}[0]{{\rm logit}}
\newcommand{\sign}[0]{{\rm sign}}

\renewcommand{\labelitemii}{$\bullet$}

\newcommand{\HAT}[1]{{\widehat{#1}}}
\newcommand{\TIL}[1]{{\widetilde{#1}}}
\newcommand{\tild}[0]{{\TIL{d}}}
\newcommand{\tile}[0]{{\TIL{e}}}
\newcommand{\tilg}[0]{{\TIL{g}}}
\newcommand{\tilu}[0]{{\TIL{u}}}
\newcommand{\tilx}[0]{{\TIL{x}}}
\newcommand{\tilP}[0]{{\TIL{P}}}
\newcommand{\tilPT}[0]{{\TIL{P}_\theta}}

\newcommand{\maparrow}[1]
{\xymatrix{\ar[r]_{#1}&}}


\newcommand{\ucalm}[0]{\ul{\calm}}

\newcommand{\bool}[0]{\{0,1\}}

\newcommand{\argmin}{\mathop{\mathrm{argmin}}\limits}
\newcommand{\argmax}{\mathop{\mathrm{argmax}}\limits}

\newcommand{\softmax}[0]{{\rm softmax}}

\newcommand{\A}[0]{\wedge}
\newcommand{\V}[0]{\vee}
\newcommand{\xor}{\oplus}
\newcommand{\bigA}[0]{\bigwedge}
\newcommand{\bigV}[0]{\bigvee}
\newcommand{\bigxor}{\bigoplus}


\newcommand{\rdart}[0]{\Rightarrow}
\newcommand{\ldart}[0]{\Leftarrow}
\newcommand{\rveps}[0]{\ul{\eps}}

\newcommand{\hatvar}[0]{\widehat{\sigma^2}}
\newcommand{\ptp}[0]{{(t)}}

\newcommand{\tseries}[1]{{\{#1\}_{\forall t}}}

\newcommand{\xbeta}[0]{X_\s^T\beta}
\newcommand{\xtau}[0]{X_\s^T\tau}

%arguments phi1, phi2, phi3, e
\newcommand{\rbd}[4]{
\xymatrix@-1.3pc{
&#1\ar[d]&&#2\ar[d]
\\
&\rvx_1\ar[rr]
&&
\rvx_2
\ar `r[rd][rd]
\\
0\ar`u[u][ru]
\ar`d[dr][rrd]
&&&&\rvA\ar[r]&#4
\\
&&
\rvx_3
\ar `r[rru][rru]
\\
&&#3\ar[u]
}
}

\newcommand{\rulezeroif}[0]{
If $(\rvb. \perp \rva.
|\rvr., \rvs.)$
in $G$, then}

\newcommand{\rulezerothen}[0]{
$\rva.=a. \leftrightarrow 1$}

\newcommand{\rulezeroP}[0]{
P(b.|a.,r.,s.)=P(b.|r., s.)}

\newcommand{\rulezeroH}[0]{
H(\rvb.:\rva.|\rvr.,\rvs.)=0}

\newcommand{\rulezeropic}[0]{
\xymatrix@C=1pc@R=1pc{
&r.\ar[dr]
&s.\ar[d]
\\
a.\ar[rr]
&&b.
&=
}
\xymatrix@C=1pc@R=1pc{
&r.\ar[dr]&s.\ar[d]
\\
a.\ar[rr]|0
&&b.
}}

\newcommand{\ruleoneif}[0]{
If $(\rvb. \perp \rva.
|\rvr., \rvs.)$
in $\cald_{\rvr.} G$, then}

\newcommand{\ruleonethen}[0]{
$\rva.=a. \leftrightarrow 1$}

\newcommand{\ruleoneP}[0]{
P(b.|a., \cald\rvr.=r.,s.)=
P(b.|\cald\rvr.=r., s.)}

\newcommand{\ruleoneH}[0]{
H(\rvb.:\rva.|\cald\rvr.,\rvs.)=0}

\newcommand{\ruleonepic}[0]{
\xymatrix@C=1pc@R=1pc{
&\cald\rvr.=r.\ar[dr]
&s.\ar[d]
\\
a.\ar[rr]
&&b.
&=
}
\xymatrix@C=1pc@R=1pc{
&\cald\rvr.=r.\ar[dr]
&s.\ar[d]
\\
a.\ar[rr]|0
&&b.
}}

\newcommand{\ruletwoif}[0]{
If $(\rvb.\perp \rva. |
 \rvr., \rvs.)$
in $\call_{\rva.}\cald_{\rvr.} G$,
 then}

\newcommand{\ruletwothen}[0]{
$\cald \rva.=a. \leftrightarrow \rva.=a.$}

\newcommand{\ruletwoP}[0]{
P(b.|\cald\rva.=a., \cald\rvr.=r., s.)=
P(b.|a., \cald\rvr.=r.,  s.)}

\newcommand{\ruletwoH}[0]{
H(\rvb.:\cald\rva.|\cald\rvr.,  \rvs.)
=
H(\rvb.:\rva.|\cald\rvr.,  \rvs.)}

\newcommand{\ruletwopicA}[0]{
\xymatrix@C=1pc@R=1pc{
\;\ar[d]_0
&\cald\rvr.=r.\ar[dr]
&s.\ar[d]
\\
\cald\rva.=a.\ar[rr]\ar[d]
&&b.
&=\quad\quad
\\
&
}
\xymatrix@C=1pc@R=1pc{
\;\ar[d]_0
&&\cald\rvr.=r.\ar[dr]
&s.\ar[d]
\\
\cald\rva.=a.\ar[d]
&a.\ar[rr]
&&b.
\\
&
}}
\newcommand{\ruletwopicB}[0]{
\xymatrix@C=1pc@R=1pc{
\;\ar[d]
&\cald\rvr.=r.\ar[dr]
&s.\ar[d]
\\
a.\ar[rr]\ar[d]
&&b.
&=\quad\quad
\\
&
}
\xymatrix@C=1pc@R=1pc{
\;\ar[d]
&&\cald\rvr.=r.\ar[dr]
&s.\ar[d]
\\
a.\ar[d]
&\cald\rva.=a.\ar[rr]
&&b.
\\
&
}}

\newcommand{\rulethreeif}[0]{
If $
(\rvb. \perp \rva.
| \rvr., \rvs.)$
in $\cald_{\rva.-an(\rvs.)}
\cald_{\rvr.}G$,
 then}

\newcommand{\rulethreethen}[0]{
$\cald \rva.=a. \leftrightarrow 1$}

\newcommand{\rulethreeP}[0]{
P(b.|\cald\rva.=a.,\cald\rvr.=r.,  s.)=
P(b.|\cald\rvr.=r., s.)}

\newcommand{\rulethreeH}[0]{
H(\rvb.:\cald\rva.|\cald\rvr., \rvs.)=0}

\newcommand{\rulethreepic}[0]{
\xymatrix@C=1pc@R=1pc{
&
&\cald\rvr.=r.\ar[dr]
&s.\ar[d]
\\
&\cald\rva.=a.\ar[rr]
&&b.
&=
}
\xymatrix@C=1pc@R=1pc{
&
&\cald\rvr.=r.\ar[dr]
&s.\ar[d]
\\
&\cald\rva.=a.\ar[rr]|0
&&b.
}}

\newcommand{\bdoordef}[0]{
Suppose that we have access to data
that allows us to
estimate a probability
distribution
 $P(x., y., z.)$.
Hence, the variables
$\rvx., \rvy., \rvz.$ are
ALL observed (i.e, not hidden).
Then we say that the
backdoor $\rvz.$
satisfies the
{\bf backdoor adjustment criterion}
relative to $(\rvx., \rvy.)$
if
\begin{enumerate}
\item
All backdoor paths from $\rvx.$
to $\rvy.$
 are blocked by  conditioning on
 $\rvz.$.
\item
$\rvz. \cap de(\rvx.)=\emptyset$.
\end{enumerate}
}

\newcommand{\bdoorclaim}[0]{
If $\rvz.$ satisfies the
backdoor criterion relative to
 $(\rvx., \rvy.)$, then

\beqa
P(y. | \cald \rvx.=x.)&=&
\sum_{z.} P(y.|x., z.)P(z.)
\\
&=&
\begin{array}{l}
\\
\\
\end{array}
\xymatrix{
\sum z.\ar[dr]
\\
x.\ar[r]
&y.
}
\eeqa
where $\sum z.$ means node
$\rvz.$ is summed over.
}

\newcommand{\fdoordef}[0]{
Suppose that we have access to data
that allows us to
estimate a probability
distribution
 $P(x., m., y.)$.
Hence, the variables
$\rvx., \rvm., \rvy.$ are
ALL observed (i.e, not hidden).
Then we say that
the frontdoor $\rvm.$
satisfies the
{\bf frontdoor adjustment criterion}
relative to $(\rvx., \rvy.)$
if
\begin{enumerate}
\item
All directed paths from
$\rvx.$ to $\rvy.$ are intercepted by
(i.e., have a node in) $\rvm.$.
\item
All backdoor paths from $\rvx.$ to
$\rvm.$ are blocked.
\item
All backdoor paths from
on $\rvm.$ to $\rvy.$
are blocked by conditioning
on  $\rvx.$.
\end{enumerate}
}

\newcommand{\fdoorclaim}[0]{
If $\rvm.$ satisfies the
frontdoor criterion
relative to $(\rvx., \rvy.)$,
and $P(x.,m.)>0$, then

\beqa
P(y. | \cald \rvx.=x.)&=&\sum_{m.}
\underbrace{\left[\sum_{x'.}
P(y.|x'., m.)P(x'.)\right]}_
{P(y.|\cald \rvm.=m.)}
\underbrace{P(m.|x.)}_
{P(m.|\cald \rvx.=x.)}
\\
&=&
\xymatrix{
&\sum x'.\ar[dr]
\\
x.´\ar[r]
&\sum m.\ar[r]&y.
}
\eeqa
where $\sum x'.$ and
$\sum m.$
means nodes
$\rvx'.$ and $\rvm.$
are summed over.
}


%Symmetry
\newcommand{\symrule}[0]{
$\rva\perp_P\rvb\implies \rvb\perp_P\rva$}

\newcommand{\symruleH}[0]{
$H(\rva:\rvb)=0\implies H(\rvb:\rva)=0$}

%Decomposition
\newcommand{\decrule}[0]{
$\rva\perp_P\rvb, \rvc\implies
\rva\perp_P\rvb \text{ and } \rva\perp_P\rvc$}

\newcommand{\decruleH}[0]{
$H(\rva:\rvb, \rvc)=0\implies
H(\rva:\rvb)=0 \text{ and } H(\rva:\rvc)=0$}

%Weak Union
\newcommand{\wearule}[0]{
$\rva\perp_P \rvb, \rvc \implies
\rva\perp_P\rvb|\rvc\text{ and }\rva\perp_P\rvc|\rvb$}

\newcommand{\wearuleH}[0]{
$H(\rva:\rvb, \rvc)=0 \implies
H(\rva:\rvb|\rvc)=0\text{ and }H(\rva:\rvc|\rvb)=0$}

%Contraction
\newcommand{\conrule}[0]{
$\rva\perp_P\rvb|\rvc\text{ and }\rva\perp_P \rvc
\implies \rva\perp_P \rvb, \rvc$}

\newcommand{\conruleH}[0]{
$H(\rva:\rvb|\rvc)=0\text{ and }H(\rva:\rvc)=0
\implies H(\rva:\rvb, \rvc)=0$}

%Intersection
\newcommand{\intrule}[0]{
$\rva\perp_P\rvb|\rvc, \rvd\text{ and }
\rva\perp_P \rvd|\rvc, \rvb\implies
\rva\perp_P \rvb,\rvd|\rvc$}

\newcommand{\intruleH}[0]{
$H(\rva:\rvb|\rvc, \rvd)=0\text{ and }
H(\rva:\rvd|\rvc, \rvb)=0\implies
H(\rva:\rvb,\rvd|\rvc)=0$}

\newcommand{\dotbarmu}[0]{{\cdot|\mu}}
\newcommand{\dotmu}[0]{{\cdot, \mu}}
\newcommand{\kbarmu}[0]{{k|\mu}}
\newcommand{\kmu}[0]{{k,\mu}}
\newcommand{\plusbarmu}[0]{{+|\mu}}
\newcommand{\plusmu}[0]{{+,\mu}}

\newcommand{\bnlearn}[0]{{\tt bnlearn\;}}

\newcommand{\sqsig}[0]{{[\sigma]}}

\newcommand{\misscellone}[0]{
\begin{array}{c}
\frac{1}{nsam}
P(x_0=0, x_2=0\cond x_1=1, \theta)
\\
\frac{1}{nsam}
P(x_0=0, x_2=1\cond x_1=1, \theta)
\\
\frac{1}{nsam}
P(x_0=1, x_2=0\cond x_1=1, \theta)
\\
\frac{1}{nsam}
P(x_0=1, x_2=1\cond x_1=1, \theta)
\end{array}
}

\newcommand{\misscelltwo}[0]{
\begin{array}{c}
\frac{1}{nsam}
P(x_1=0\cond x_0=0,x_2=1, \theta)
\\
\frac{1}{nsam}
P(x_1=1\cond x_0=0,x_2=1,  \theta)
\end{array}
}


\newcommand{\td}[0]{{\TIL{d}}}
\newcommand{\rvtd}[0]{{\ul{\TIL{d}}}}
\newcommand{\tx}[0]{{\TIL{x}}}
\newcommand{\tmu}[0]{{\TIL{\mu}}}
\newcommand{\rvtx}[0]{{\ul{\TIL{x}}}}

\newcommand{\mlarr}[0]{\xrightarrow{\rm ML-fit}}
\newcommand{\lrarr}[0]{\xrightarrow{\rm LR-fit}}

\newcommand{\setprob}[3]
{{\begin{array}{c}S=\{#1\}
\\P(S)=#2\\ \haty(x^\s_S)=\$#3 K
\end{array}}}

\newcommand{\Gno}[0]{\xymatrix{\;\ar[r]|\parallel_G&}}
\newcommand{\Gyes}[0]{\xymatrix{\;\ar[r]_G&}}

\newcommand{\calypso}[0]{\ol{\caly}}

\newcommand{\SeqBdoorDef}[0]{
Suppose that we have access to data
that allows us to
estimate a probability
distribution
 $P(x^n, y, z^n)$.
Hence, the variables
$\rvx^n, \rvy, \rvz^n$ are
ALL observed (i.e, not hidden).
Then we say that the
the multinode
of ``covariates" $\rvz^n$
satisfies the
{\bf sequential backdoor (SBD) adjustment criterion}
relative to $(\rvx^n, \rvy)$
if for all $t\in\{0,1, \ldots, n-1\}$,

\begin{enumerate}
\item
$\rvy\perp\rvx_t|
\underbrace{(\rvx_0, \rvx_1, \ldots,\rvx_{t-1},
\rvz_0, \rvz_1, \ldots, \rvz_t)}
_{\text{Past of $\rvx_t$}}$
in $\call_{\rvx_{t}}
\cald_{\rvx_{t+1},\rvx_{t+2}
,\ldots, \rvx_{n-1}}G$.
\item
$\rvz_t \cap de(\rvx_t)=\emptyset$.
\end{enumerate}
}



\newcommand{\SeqBdoorClaim}[0]{
If $\rvz^n$ satisfies the
sequential backdoor criterion relative to
 $(\rvx^n, \rvy)$, then

\beq
P(y | \cald \rvx^n=x^n)=
\calq(y|x^n)
\;,
\eeq
where $\calq(y|x^n)$
is defined by
Eq.(\ref{def-q-y-xn-seqbdoor}).
}

%https://tex.stackexchange.com/questions/208905/loops-of-different-sizes

\newcommand{\loopup}[2]{ % \ar@(ul,ur) size is like 3
\ar@`{[]+/ul+#1pc/,[]+/ur+#1pc/}#2[]}
\newcommand{\loopdown}[2]{ % \ar@(ul,ur) size is like 3
\ar@`{[]+/dl+#1pc/,[]+/dr+#1pc/}#2[]}
\newcommand{\loopright}[2]{ % \ar@(ul,ur) size is like 3
\ar@`{[]+/dr+#1pc/,[]+/ur+#1pc/}#2[]}
\newcommand{\loopleft}[2]{ % \ar@(ul,ur) size is like 3
\ar@`{[]+/dl+#1pc/,[]+/ul+#1pc/}#2[]}

% test code
%\xymatrix{
%\rva\loopup{3}{@[green]_r}
%&
%\rvb\loopdown{3}{@[red]_r}
%&
%\rvc\loopright{3}{@[blue]_r}
%\\
%\rvd\loopleft{3}{@[violet]_r}
%}

\usepackage[T1]{fontenc}
\usepackage{underscore}
\begin{document}

%        Assume:
%        batch_size= 24, s_{ba}
%        hidden_size= 768,  d
%        NUM_ILABELS= 6, n_{il}
%        ILABELLING_DIM= 30
%        \Lam=2 iterative layers 
%		D=5 number of depths.
%
%        Below we show the shape of the input and output tensors for each layer.
%
%        LINES for depth=0
%        LINES for depth=1
%        LINES for depth=2
%        LINES for depth=3
%        LINES for depth=4
%
%        where LINES=
%        encoding_layer: [24, 84, 6]->[24, 105, 768]
%        *****iterative layer 0: [24, 105, 768]->[24, 105, 768]
%        dropout: [24, 105, 768]->[24, 105, 768]
%        bunch of torch operations: [24, 105, 768]->[24, 84, 768]
%        merge layer: [24, 84, 768]->[24, 84, 300]
%        ilabelling_layer: [24, 84, 300]->[24, 84, 6]
%        encoding_layer: [24, 84, 6]->[24, 105, 768]
%        *****iterative layer 1:  [24, 105, 768]->[24, 105, 768]
%        dropout: [24, 105, 768]->[24, 105, 768]
%        bunch of torch operations: [24, 105, 768]->[24, 84, 768]
%        merge layer: [24, 84, 768]->[24, 84, 300]
%        ilabelling_layer: [24, 84, 300]->[24, 84, 6]


SentenceAx (at github repo Ref.\cite{sentence-ax-github}) is a complete re-write of the Openie6 software 
(at github repo Ref.\cite{openie6-repo}).
 The Openie6 software is described by its creators
 in the paper Ref.\cite{openie6-paper},
 which we will henceforth refer to as O6.
 
 This chapter describes the technical
 aspects of SentenceAx. Although this chapter
 can be read without reading O6, we highly recommend to
 our readers that they read O6 also. 
 Some parts of this chapter are taken almost verbatim
 from O6. Other parts try to fill-in gaps in the
 explanations provided by O6 or to improve those explanations. Yet others parts describe tiny changes that we made to the Openie6 software, in an effort to improve its clarity. 
 
 
 
 This chapter often uses different
 notation from that in O6, because the author
 is a compulsive notation changer. The fact that I
 might use a different notation to say the same thing by no means should be
 interpreted  as an affirmation that I believe I have improved the O6 theory in a non-trivial way. I haven't.
 
 In this chapter, we 
 will use the Numpy-like tensor notation
 discussed in Section 
 \ref{sec-numpy-tensors}. In particular, note that $[n] = [0:n] = \{0, 1,\ldots, n-1\}$ and that $T^{[n], [m]}$ is an $n\times m$ matrix.

\section{Preliminary Conventions}

\subsection{Tensor Notation}
Our tensor notation is discussed in Section 
\ref{sec-numpy-tensors}.
Here is a quick review
of some of the more essential
facts in that section on tensors.
Below will often accompany 
  an equation in tensor
  component notation
  with, in parenthesis, the equivalent matrix equation.
 
 

\begin{itemize}

\item{\bf reshaping}

\beq
T^{\nu, \delta}\rarrow T^{\Delta}
\;\;
\left(
T^{[n_\rvh], [d]} \rarrow T^{[D]}
\right)
\eeq

\beq
T^{\Delta}\rarrow T^{\nu, \delta}
\;\;
\left(
T^{[D]}\rarrow T^{[n_\rvh], [d]}
\right)
\eeq

\item {\bf concatenation}
\beq
T^{[n]}= (T^0, T^1,\ldots, T^{n-1})= 
(T^\nu)_{\nu\in[n]}
\eeq

\item {\bf Hadamard product (element-wise, entry-wise multiplication)}
\beq
T^{[n]}* S^{[n]}= (T^\nu S^\nu)_{\nu\in[n]}
\eeq


\item {\bf Matrix multiplication}

$T^{[n]}= T^{[n], [1]}$ is a column vector.

\beq
(T^{[n]})^T S^{[n]}=\text{scalar}
\eeq

\beq
T^{[a],[b]}S^{[b], [c]}
=\left[\sum_{\beta\in[b]} T^{\alp, \beta}
S^{\beta, \gamma}\right]
_{\alp_\in [a], \gamma \in [c]}
\eeq
Most treatments of tranets, including the
O6 and PyTorch,  order the
operations chronologically from
left to right. So if $A$ occurs before $B$,
they write $AB$.
This is contrary 
to what is done in Linear Algebra, where one 
orders the operations chronologically from right to left, and one writes $BA$.
We will adhere to the Linear Algebra
convention, since it is so prevalent
and is the overwhelming precedent.
\end{itemize}


\subsection{PyTorch conventions}



\begin{itemize}
\item {\tt output = nn.Linear(na, nb)(input)}

\beq
x^{[na]}\rarrow y^{[nb]}=W^{[nb], [na]}x^{[na]}
\eeq

\item{\tt embedding = nn.Embedding(num_embeddings=L, embedding_dim=d)}

Let

$L=$ {\tt num_embeddings}

$d=$ {\tt embedding_dim}


$\lam\in[L]$, $\alp\in[\ell]$,
$\nu_1\in [n_1]$, $\nu_2\in[n_2]$ 

$\ell = \nu_1 \nu_2$.

Consider a matrices $X, E, Y$ such that

\beq
Y^{\delta, \alp} = \sum_{\lam}E^{\delta, \lam}
X^{\lam, \alp}
\;\;\;
\left(
Y^{[d], [\ell]} = E^{[d], [L]} X^{[L], [\ell]}
\right)
\eeq
Assume, furthermore, that 
matrix $X$ has 1-hot columns

\beq
X^{\lam, \alp}
=
\delta(\lam, \lam(\alp))
\eeq
where $\lam(): [\ell]\rarrow [L]$.

Hence,

\beq
Y^{\delta, \alp} = E^{\delta, \lam(\alp)}
\eeq


\beq
{\ul \lam}^{\alp} =\lam(\alp)
\eeq

\beq
{\ul \lam}^\alp\rarrow Y^{\delta, \alp}=E^{\delta, \lam(\alp)}
\;\;
(
{\ul \lam}^{[\ell]}
\rarrow Y^{[d], [\ell]})
\eeq

Replace $\alp$ by 
$(\nu_1, \nu_2)$.

\beq
{\ul \lam}^{\nu_1, \nu_2}\rarrow Y^{\delta, \nu_1, \nu_2}=E^{\delta, \lam(\nu_1, \nu_2)}
\;\;
(
{\ul \lam}^{[n_1], [n_2]}
\rarrow Y^{[d], [n_1], [n_2]})
\eeq
Actually, {\tt nn.Embedding()} orders the tensor
indices of the output thus

\beq
{\ul \lam}^{[n_1], [n_2]}
\rarrow Y^{[n_1], [n_2], [d]}
\eeq
so the $[d]$ is attached on the right side rather than the left one.

\item{\bf Cross Entropy Loss}

{\bf Cross Entropy}  
in Information Theory

\beqa
H(P_{tar}^\s, P_{in}^\s)
&=&
-\sum_{\gamma\in[n_\rvc]}P_{tar}(\gamma|\s) \ln P_{in}(\gamma|\s)
\\
&=&-\sum_{\gamma\in[n_\rvc]}P_{tar}(\gamma|\s) \ln 
\left[\frac{P_{in}(\gamma|\s)}{P_{tar}(\gamma|\s)}
P_{tar}(\gamma|\s)\right]
\\
&=&
H(P_{tar}^\s) + D_{KL}(P_{in}^\s\parallel P_{tar}^\s)
\eeqa
{\bf Cross Entropy Loss} in PyTorch

Let

$n_\rvs=$ total number of samples being considered,
usually batch size.
$\s\in [n_\rvs]$

$n_\rvc=$ number of classes in classification. $\gamma\in[n_\rvc]$


$x^{[n_\rvc], [n_\rvs]}=$ input  samples

$t^{[n_\rvs]}=$ target samples

\beqa
P_{in}(\gamma|\s)&=&
\frac{\exp(x^{\gamma, \s})}
{\sum_{\gamma'\in[n_\rvc]}\exp(x^{\gamma', \s})}
\\
&=&
{\rm softmax}(x^{[n_\rvc], \s})(\gamma|\s)
\eeqa

Suppose $W^\gamma:values(\rvt)\rarrow[0,1]$
for all $\gamma\in[n_\rvc]$.

\beq
P_{tar}(\gamma|\s)=
\frac{
W^\gamma (t^{\s})\indi(t^{\s}\neq -100)}
{\sum_{\gamma\in[n_\rvc]}numerator}
\eeq

-100 can be replaced by any other integer 
in $values(\rvt)$ for which we want the loss to be zero (for example, an integer used for padding)




\beq
\call_{CE}^\s(t^{\s}, x^{[n_\rvc], \s})=
H(P_{tar}(\cdot|\s), P_{in}(\cdot|\s))
\eeq


\beq
\call_{CE} = \frac{1}{n_\rvs} \sum_{\s\in[n_\rvs]}\call_{CE}^\s
\eeq

For example, if $W^\gamma=1$, and $n_\rvc=2$,

\beqa
\call_{CE} &=& \frac{1}{n_\rvs}\sum_{\s}
\left[
P_{tar}(0|\s)\ln P_{in}(0|\s)
+
P_{tar}(0|\s)\ln (1-P_{in}(0|\s))
\right]
\eeqa

In SentenceAx

$x^{[n_\rvc], [n_\rvs]}=$ {\tt ll_loss_input}

$t^{[n_\rvs]}$ = {\tt l_loss_target}


\end{itemize}

\section{Bayesian Network for this model}

Let

$\ell_{pad}=84$, padding length, for this batch

$\ell_{enc}=105$, encoded length, for this batch, $\ell_{enc}\geq \ell_{pad}$

$n_{dep}=5$, number of copies of solid box connected in series, number of depths

 $n_{att}=2$, number of copies of
dashed box connected in series, number of iterative (attention) layers.


$d=768$, hidden dimension per head

$n_\rvh$, number of heads (BERT base)

$D=d n_\rvh$, hidden dimension
for all heads


$s_{ba}=24$, batch size

$n_{ila}=6$, number of ilabels

$d_{ila}=300$, ilabeling dimension

\section{Loss $\call$ for this model}

The Loss $\call$ is the sum of the
Cross Entropy Loss $\call_{CE}$ and 4 penalty losses $\call_i$ for $i\in PL$ where 
$PL=\{ POSC, HVC, HVE, EC\}$.


\beq
\call = \call_{CE} + 
\sum_{i\in PL} \lam_{i}\call_i
\eeq

The $\lam_i$ are hyper-parameters.

Next we will define the CE loss and the 4 penalty losses.





\subsection{Penalty Losses}

Below, we will use the standard 
notation for the positive-part function (a.k.a. 
the RELU function)

\beqa
(x)_+ &=&
\left\{
\begin{array}{ll}
x& \text{ if } x\geq0
\\
0& \text{ if } x< 0
\end{array}
\right.
\\
&=& {\rm max}(0, x)
\eeqa
Since loss is supposed to be bounded below
 (usually loss is greater or equal to zero),
 the positive-part function arises frequently  when defining a loss.



Let

$\ell=$ number of words, length of sentence. $\alp \in [\ell]$

$M=$ number of depths. $\mu\in[M]$

$w^\alp=$ word at position $\alp$

$T_{pos}=\{N, V, JJ, RB\}$, POS tags, N=Noun, V=Verb, JJ=Adjective, RB=Adverb

$T_{ex}=\{ N, R, O, S\}$\footnote{
The Openie6 and SentenceAx software use 
a different set for $T_{ex}$. In SentenceAx, we use for
$T_{ex}$ the list {\tt BASE_EXTAGS} (defined globally 
in the file {\tt sax_globals}.)
In {\tt BASE_EXTAGS}, N becomes NONE (or 0)
and R becomes REL (or 3).
 
Also, 2 tranets are trained by Openie6 and SentenceAx,
one for extraction (task=ex), and one for splitting (task=cc).
For splitting, $T_{ex}$ is replaced by $T_{cc}$. In SentenceAx, we use for
$T_{cc}$ the list {\tt BASE_CCTAGS}  (defined globally 
in the file {\tt sax_globals}.)
In {\tt BASE_CCTAGS}, N becomes NONE (or 0)
and R becomes CC (or 3).}. extraction tags (extags),  S=Subject, R=Relation, O=Object, N=None


$T_{ex}\backslash N = T_{ex}-\{N\}$


$POS^\alp\in T_{pos}$, Part Of Sentence of $w^\alp$. 



Importance indicator function. 
\beq
IMP^\alp = \indi(POS^\alp \in T_{pos})
\eeq

Head verb indicator function. A {\bf head verb} is any verb that isn't a {\bf light verb}
(do, be, is, has, etc.).

\beq
HV^\alp = \indi(w^\alp \text{ is a head verb})
\eeq

We will
derive from the data an empirical probability $P(\rvt^{\mu, \alp} =t )$ for a table element
$\rvt^{\mu, \alp}\in T_{ex}$, for all $\mu\in [M]$ and $\alp\in[\ell]$.

O6 describes how the 4 penalty losses 
deal with the following sentence, which we will refer to
henceforth as the red-sent:

\begin{quote}\color{red}
Obama gained popularity after
Oprah endorsed him for the presidency.
\end{quote}

For the red-sent, the head verbs are {\color{red} gained, endorsed}

Two valid extractions from red-sent are:
{\color{red}(Obama; gained; popularity)}
and {\color{red}(Oprah; endorsed him for;
 the presidency)}.


\begin{enumerate}

\item {\bf Part of Speech Coverage (POSC)}

All words important words should be part of at least one extraction.

In red-sent: the words {\color{red}Obama, gained, popularity,
Oprah, endorsed, presidency} must be covered in
the set of extractions.

\beq
\call_{POSC}=\sum_{\alp\in [\ell]}IMP^{\alp}P_{POSC}(\alp)
\eeq

\beq
P_{POSC}(\alp)=
1-{\rm max}_{\mu\in [M]}
{\rm max}_{t\in T_{ex}\backslash N}P(\rvt^{\mu, \alp}=t)
\eeq

\item {\bf Head Verb Coverage (HVC)}

Each head verb
should be present in the relation (R) span of some
(but not too many) extractions.\footnote{A span is a list of contiguous words.}

In red-sent: {\color{red} (Obama;
gained; popularity), (Obama; gained; presidency)} is not a comprehensive set of extractions.

\beq
\call_{HVC}=
\sum_{\alp\in [\ell]}
HV^\alp P_{HVC}(\alp)
\eeq

\beq
P_{HVC}(\alp)=
\left|
1-\sum_{\mu\in [M]}P(\rvt^{\mu, \alp}=R)
\right|
\eeq

\item {\bf Head Verb Exclusivity (HVE)}

The relation (R) span
of one extraction can contain at most one head
verb.

In red-sent: {\color{red}gained popularity after Oprah endorsed} is not a good relation as it contains two
head verbs

\beq
\call_{HVE}=
\sum_{\mu\in [M]}
\left(
\sum_{\alp\in [\ell]}
HV^\alp P(\rvt^{\mu, \alp}=R)
-1
\right)_+
\eeq

\item {\bf Extraction Count (EC)}

The total number of extractions with head verbs in the relation (R) span
must be no fewer than the number of head verbs
in the sentence.

\beq
\call_{EC}=
\left(
\sum_{\alp\in[\ell]} HV^\alp
-
\sum_{\mu\in [M]}EC^\mu
\right)_+
\eeq

\beq
EC^\mu=
{\rm max}_{\alp\in [\ell]}
HV^\alp P(\rvt^{\mu, \alp}=R)
\eeq


\end{enumerate}






\end{document}