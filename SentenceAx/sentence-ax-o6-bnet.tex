\documentclass[12pt]{article}
\input{bayesuvius.sty}
\begin{document}

\begin{figure}[h!]\centering
$$\xymatrix@R=2.5pc@C=3.5pc{
&&&
\\
&&&
\\
*+[F*:pink]{\underline{S}^{[86], [768]}}\ar@[red][rr]|-{\color{red} depth\neq 0}&&*+[F*:SpringGreen]{\underline{M}^{[86], [300]}}\ar[r]_{W_{il}}&*+[F*:SkyBlue]{\underline{L}^{[86], [6]}}
\\
&*+[F*:pink]{\underline{E}^{[86], [768]}}\ar[ul]|-{1}&*+[F*:yellow]{\underline{a}^{[86]}}\ar[l]&
\\
&*+[F*:pink]{\underline{G}^{[86], [768]}}\ar[ur]\ar@[green]@/_1pc/[uur]|-{\color{green} depth=0}\ar[uul]|-{1}&*+[F*:pink]{\underline{d}^{[121], [768]}}\ar[uuuul]\ar[l]&
\\
&*+[F*:Orchid]{\underline{n}^{[121], [768]}}\ar[ur]&&
\\
&*+[F*:Dandelion]{\underline{A}^{[121], [D]}}\ar[u]_{W_\rva}&&
\\
*+[F*:Dandelion]{\underline{V}^{[121], [D]}}\ar[ur]&*+[F*:Dandelion]{\underline{K}^{[121], [D]}}\ar[u]&*+[F*:Dandelion]{\underline{Q}^{[121], [D]}}\ar[ul]&
\\
&&&
\\
&*+[F*:Orchid]{\underline{B}^{[121], [768]}}\ar[uu]|-{W_\rvk}\ar[uur]|-{W_\rvq}\ar[uul]|-{W_\rvv}&&
\save
\POS"3,1"."9,1"."3,3"."9,3"!C*+<4.8em>\frm{-,}
\POS"6,1"."9,1"."6,3"."9,3"!C*+<3.8em>\frm{--}
\restore
}$$
\caption{SentenceAx Bayesian network. 2 copies of dashed box are connected in series. 5 copies of plain box are connected in series. We display the tensor shape superscripts in the Linear Algebra R2L order. (PyTorch uses a L2R order instead). All tensor shape superscripts have been simplified by omitting a $[s_{ba}]$, where $s_{ba}=24$ is the batch size. $D= d n_\rvh$ where $d=768$ is the hidden dimension per head, and $n_\rvh=12$ is the number of heads. }
\label{fig-texnn-for-sentence-ax-o6-bnet}
\end{figure}

\begin{tabular}{ll}
$\underline{a}^{[86]}$ :&{\tt ll\_greedy\_ilabel}\\
$\underline{B}^{[121], [768]}$ :&{\tt lll\_hidstate}\\
$\underline{d}^{[121], [768]}$ :&{\tt lll\_hidstate}\\
$\underline{E}^{[86], [768]}$ :&{\tt lll\_pred\_code}\\
$\underline{G}^{[86], [768]}$ :&{\tt lll\_word\_hidstate}\\
$\underline{L}^{[86], [6]}$ :&{\tt lll\_word\_score}\\
$\underline{M}^{[86], [300]}$ :&{\tt lll\_word\_hidstate}\\
$\underline{n}^{[121], [768]}$ :&{\tt lll\_hidstate}\\
$\underline{S}^{[86], [768]}$ :&{\tt lll\_word\_hidstate}
\end{tabular}



\begin{subequations}

\begin{equation}\color{blue}
A^{[121], [D]} = \text{Attention}(Q^{[121], [D]},K^{[121], [D]},V^{[121], [D]})
\label{eq-A-fun-sentence-ax-o6-bnet}
\end{equation}

\begin{equation}\color{blue}
\begin{aligned}
a^{[86]} &= \text{argmax}(G^{[86], [768]};dim=-1)
\label{eq-a-fun-sentence-ax-o6-bnet}
\\ &:{\tt ll\_greedy\_ilabel}
\end{aligned}
\end{equation}

\begin{equation}\color{blue}
\begin{aligned}
B^{[121], [768]} &= \text{BERT}()
\label{eq-B-fun-sentence-ax-o6-bnet}
\\ &:{\tt lll\_hidstate}
\end{aligned}
\end{equation}

\begin{equation}\color{blue}
\begin{aligned}
d^{[121], [768]} &= \text{dropout}(n^{[121], [768]})
\label{eq-d-fun-sentence-ax-o6-bnet}
\\ &:{\tt lll\_hidstate}
\end{aligned}
\end{equation}

\begin{equation}\color{blue}
\begin{aligned}
E^{[86], [768]} &= \text{embedding}(a^{[86]})
\label{eq-E-fun-sentence-ax-o6-bnet}
\\ &:{\tt lll\_pred\_code}
\end{aligned}
\end{equation}

\begin{equation}\color{blue}
\begin{aligned}
G^{[86], [768]} &= \text{gather}(d^{[121], [768]};dim=-2)
\label{eq-G-fun-sentence-ax-o6-bnet}
\\ &:{\tt lll\_word\_hidstate}
\end{aligned}
\end{equation}

\begin{equation}\color{blue}
K^{[121], [D]} = B^{[121], [768]}W_\rvk^{[768], [D]}
\label{eq-K-fun-sentence-ax-o6-bnet}
\end{equation}

\begin{equation}\color{blue}
\begin{aligned}
L^{[86], [6]} &= M^{[86], [300]}W_{il}^{[300],[6]}
\label{eq-L-fun-sentence-ax-o6-bnet}
\\ &:{\tt lll\_word\_score}
\end{aligned}
\end{equation}

\begin{equation}\color{blue}
\begin{aligned}
M^{[86], [300]} &= G^{[86], [768]}W_{il}^{[768], [300]}
\label{eq-M-fun-sentence-ax-o6-bnet}
\\ &:{\tt lll\_word\_hidstate}
\end{aligned}
\end{equation}

\begin{equation}\color{blue}
\begin{aligned}
n^{[121], [768]} &= A^{[121], [D]}W_\rva^{[D], [768]}
\label{eq-n-fun-sentence-ax-o6-bnet}
\\ &:{\tt lll\_hidstate}
\end{aligned}
\end{equation}

\begin{equation}\color{blue}
Q^{[121], [D]} = B^{[121], [768]}W_\rvq^{[768], [D]}
\label{eq-Q-fun-sentence-ax-o6-bnet}
\end{equation}

\begin{equation}\color{blue}
\begin{aligned}
S^{[86], [768]} &= E^{[86], [768]} + G^{[86], [768]}
\label{eq-S-fun-sentence-ax-o6-bnet}
\\ &:{\tt lll\_word\_hidstate}
\end{aligned}
\end{equation}

\begin{equation}\color{blue}
V^{[121], [D]} = B^{[121], [768]}W_\rvv^{[768], [D]}
\label{eq-V-fun-sentence-ax-o6-bnet}
\end{equation}

\end{subequations}


\end{document}  
