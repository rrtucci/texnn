\documentclass[12pt]{article}
\input{bayesuvius.sty}
\begin{document}

\begin{figure}[h!]\centering
$$\xymatrix{
*+[F*:Dandelion]{\underline{PAT1}}\ar[dr]&&*+[F*:Dandelion]{\underline{P1F3}}\ar[d]\ar[dl]\ar[drr]\ar[dr]&&*+[F*:Dandelion]{\underline{APRR}}\ar[d]&&
\\
&*+[F*:Orchid]{\underline{CHS}}&*+[F*:Orchid]{\underline{CAB1}}&*+[F*:Orchid]{\underline{RBCB1A}}&*+[F*:Dandelion]{\underline{LHY}}\ar[dr]\ar[d]\ar@/_1pc/[rr]&&*+[F*:Orchid]{\underline{GI}}\ar@/_1pc/[ll]
\\
*+[F*:Orchid]{\underline{HOG1}}\ar[ur]&&*+[F*:Orchid]{\underline{STO}}\ar[ul]\ar[dr]&&*+[F*:Orchid]{\underline{GBSS1}}&*+[F*:Orchid]{\underline{CAT3}}\ar[dll]&*+[F*:Orchid]{\underline{DND1}}\ar[dlll]
\\
&&&*+[F*:Orchid]{\underline{RCD1}}&&&
}$$
\caption{Hybrid rice gene reg net}
\label{fig-texnn-for-rice-gene-reg-net}
\end{figure}

\begin{subequations}

\begin{equation}\color{blue}
APRR = 
\label{eq-A-fun-rice-gene-reg-net}
\end{equation}

\begin{equation}\color{blue}
CAB1 = P1F3
\label{eq-b-fun-rice-gene-reg-net}
\end{equation}

\begin{equation}\color{blue}
CAT3 = LHY
\label{eq-d-fun-rice-gene-reg-net}
\end{equation}

\begin{equation}\color{blue}
CHS = PAT1,P1F3,HOG1,STO
\label{eq-C-fun-rice-gene-reg-net}
\end{equation}

\begin{equation}\color{blue}
DND1 = 
\label{eq-D-fun-rice-gene-reg-net}
\end{equation}

\begin{equation}\color{blue}
GBSS1 = LHY
\label{eq-c-fun-rice-gene-reg-net}
\end{equation}

\begin{equation}\color{blue}
GI = LHY
\label{eq-G-fun-rice-gene-reg-net}
\end{equation}

\begin{equation}\color{blue}
HOG1 = 
\label{eq-H-fun-rice-gene-reg-net}
\end{equation}

\begin{equation}\color{blue}
LHY = P1F3,APRR,GI
\label{eq-L-fun-rice-gene-reg-net}
\end{equation}

\begin{equation}\color{blue}
P1F3 = 
\label{eq-a-fun-rice-gene-reg-net}
\end{equation}

\begin{equation}\color{blue}
PAT1 = 
\label{eq-P-fun-rice-gene-reg-net}
\end{equation}

\begin{equation}\color{blue}
RBCB1A = P1F3
\label{eq-R-fun-rice-gene-reg-net}
\end{equation}

\begin{equation}\color{blue}
RCD1 = STO,CAT3,DND1
\label{eq-e-fun-rice-gene-reg-net}
\end{equation}

\begin{equation}\color{blue}
STO = 
\label{eq-S-fun-rice-gene-reg-net}
\end{equation}

\end{subequations}


\end{document}  
